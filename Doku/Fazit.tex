\section{Fazit}

Zusammengenommen ist in dem Projekt das gewünschte Ergebnis erreicht worden und der \textit{einfache} Banking-Server wurde voll funktionsfähig implementiert. Alle wesentlichen Anforderungen konnten umgesetzt werden. Besonders profitieren konnte die Projekt-Gruppe A dabei vom Einsatz des Behaviour-Moduls \textit{gen\textunderscore server}. Die Umsetzung gelang derart, dass der Server auch von anderen Rechnern über den Knoten \textit{bs@localhost} angesprochen werden kann. Das Projekt erforderte eine tiefgehende Auseinandersetzung mit der Erlang-Programmierumgebung und gute Teamarbeit. Abschließend kann festgestellt werden, dass diese, oder eine ähnliche Serverstruktur, durchaus in der Programmiersprache Erlang implementiert werden kann und auch in der Wirtschaft eingesetzt werden könnte. Insbesondere die Parallelität und die Informationsvermittlung über das Senden von Nachrichten sind Aspekte, die Potential bieten und für einen Einsatz von Erlang-Serverstrukturen sprechen.