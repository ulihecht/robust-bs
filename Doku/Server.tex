\subsection{Server}
Das Server-Modul dient der Interaktion von Client und Worker-Prozessen. Das Modul nutzt die ''gen\_server''-Funktionalität, welche eine möglichst einfache Client/Server-Kommunikation ermöglicht.\\
Der Server wird mittels \texttt{bs:start()} gestartet. Diese Funktion ruft intern \texttt{gen\_server:start\_link(ServerName, CallBackModule, Arguments, Options)} auf, wie in folgendem Modulausschnitt zu sehen ist.
\begin{lstlisting}
 start() -> gen_server:start_link({local, ?MODULE}, ?MODULE, [], []).
\end{lstlisting}
Der Aufruf lässt erkennen, dass keine erweiterten Funktionalitäten des generischen Servers benutzt werden. Es wird lediglich der Name des Servers festgelegt, welcher gleich dem Modulnamen ist und der Name des Callback-Moduls, welches dem eigenen Modul entspricht. Da für die Funktionalität des Banking-Servers keine Initialisierungsdaten und Optionen benötigt werden, werden hierfür nur leere Listen übergeben.\\
Durch den Aufruf von \texttt{gen\_server:start\_link} wird die Init-Funktion des Banking-Servers ausgeführt.
Innerhalb der Init-Funktion wird das Flag ''\textit{trap\_exit}'' auf \textit{true} gesetzt. Dies führt dazu, dass vom Server aufgespannte Prozesse an ihn eine Nachricht senden, wenn sie beendet werden und er im Falle eines abgestürzten Prozesses nicht selber beendet wird. Zudem wird in der Init-Funktion die \textit{Dets}-Datenbank geöffnet, welche die offenen Transaktionen verwaltet. Geschlossen wird die Datenbank erst wieder, wenn der Server innerhalb der \texttt{terminate}-Methode beendet wird.\\
Die Hauptaktivität des Banking-Servers liegt in den Methodenaufrufen von \texttt{handle\_cast(Name, Message)}. Dies sind die Callback-Funktionen der vom Client aufgerufenen \texttt{gen\_server:cast(...)}-Methoden. Für jeden Transaktionstyp existiert dabei eine \texttt{handle\_call}-Callback-Methode. Wie diese im Detail aufgebaut sind, soll folgender Ausschnitt verdeutlichen.
\begin{lstlisting}
 handle_cast({geld_einzahlen, ClientPId, Kontonr, Verwendungszweck, Betrag}, LoopData) ->
   erzeuge_transaktion(geld_einzahlen, [ClientPId, Kontonr, Verwendungszweck, Betrag]),
   {noreply, LoopData};
\end{lstlisting}
Zu Erkennen ist, dass die Nachricht jeweils auf die auszuführende Transaktion gematched wird. Im dargestellten Fall auf ''\textit{geld\_einzahlen}''. Die Reihenfolge der Argumente ist im Wesentlichen stets die selbe. Nach der auszuführenden Aktion folgt die Client-PID, mittels derer der Worker-Prozess später eine Antwort an den Client zurückschicken kann. Danach folgt, falls nötig, die Kontonummer und anschließend weitere Argumente, die für die Ausführung der Aktion notwendig sind.\\
Innerhalb der Funktion wird die Methode \texttt{erzeuge\_transaktion(Aktion, Args)} aufgerufen, welche im folgenden Ausschnitt gezeigt wird.
\begin{lstlisting}
 erzeuge_transaktion(Action, Arg) ->
   TId = spawn_link(bw, init, []),
   dets:insert(transaction, {TId, {Action, Arg}}),
   TId ! [Action|Arg].
\end{lstlisting}
Die Funktion spannt den Worker-Prozess auf und ruft die in ihm enthaltene Init-Funktion auf. Die zurückerhaltene Prozess-ID wird zusammen mit dem Transaktionstyp und den zusätzlichen Argumenten in der Transaktionsdatenbank gespeichert. Anschließend wird an den gestarteten Worker eine Nachricht mit der auszuführenden Transaktion und den Argumenten gesendet.\\
Wie bereits erwähnt, überwacht der Server die Worker-Prozesse, und bekommt eine Nachricht, falls sich ein solcher beendet. Hierbei wird zwischen einem normalen und einem fehlerhaften Beenden unterschieden. Wie dies getan wird zeigt folgender Ausschnitt aus dem Server-Modul.
\begin{lstlisting}
 handle_info({'EXIT', TId, error}, LoopData) -> 
    io:format("Worker Exit: ~p (~p)~n", [error, TId]),
    ...
   {noreply, LoopData};

 handle_info({'EXIT', PId, normal}, LoopData) -> 
    io:format("Worker Exit (not handled): ~p (~p)~n", [normal, PId]),
    dets:open_file(transaction, [{file, "db_transaction"}, {type, set}]),
    dets:delete(transaction, PId),
    dets:close(transaction),
 {noreply, LoopData}.
\end{lstlisting}
Es ist ersichtlich, dass die Nachrichten, welche über die Beendigung eines Prozesses informieren, mit der Callback-Funktion \texttt{handle\_info(Nachricht, LoopData)} abgehandelt werden. Übergeben wird dabei in jedem Fall die Zeichenkette \textit{'EXIT'}, die Prozess-ID des beendeten Prozesses und die Art der Beendigung als Atom.\\
Ist die Beendigungsart als ''\textit{normal}'' angegeben, bedeutet dies, dass der Prozess ordnungsgemäß durchgelaufen ist. Sollte dies der Fall sein, wird aus der Transaktionsdatenbank des Servers die Transaktion, für welche der beendete Prozess zuständig war, entfernt. So stehen innerhalb der Datenbank nur Transaktionen, die noch ausgeführt werden müssen, oder gerade ausgeführt werden.\\
Ist ein Prozess ''gekillt'' worden, oder abgestürzt, lautet der Beendigungsgrund ''\textit{error}''. Da nicht klar ist, ob der Prozess zum Zeitpunkt seines Absturzes bereits die Transaktion ausgeführt hat oder nicht, muss zunächst in der Datenbank der Worker geprüft werden, ob die Transaktion auf dem entsprechenden Konto bereits gespeichert wurde. Ist dies der Fall, kann die Transaktion aus der Transaktionsliste des Servers gelöscht werden. Ansonsten muss diese neu gestartet werden.\\
Besondere Aufmerksamkeit muss man Transaktionen widmen, die zwei Konten gleichzeitig bearbeiten. Darunter fällt eine Überweisung. Hier wird von einem Konto zunächst Geld abgebucht und daraufhin auf das andere Konto eingezahlt. Ist der Worker-Prozess genau zwischen diesen zwei Schritten abgestürzt, kann die Überweisungstransaktion nicht einfach erneut ausgeführt werden, da dies dazu führen würde, dass zweimal Geld abgebucht wird. Um dies zu verhindern, kann eine Überweisung wahlweise komplett neu gestartet werden, oder im zweiten Schritt (Geld auf dem Empfängerkonto gutschreiben) fortgesetzt werden.