In diesem Dokument wird die Projektarbeit mit dem Thema Banking-Server dokumentiert. Ziel war es, einen Banking-Server mit allen gängigen Funktionen in der Programmiersprache Erlang zu implementieren. Grundlage war das \textit{Behaviour-Modul} \textbf{gen\textunderscore server}, welches hauptsächlich der Kommunikation diente. Die Bearbeitung der Aktionen und Transaktionen sollte in sogenannte \textit{Workerprozesse} ausgelagert werden. Zudem sollte es möglich sein, mit mehreren Clients gleichzeitig auf den Server zu agieren. Eine persistente Datenspeicherung, sowie eine Fehlerbehandlung bei Ausfall eines \textit{Workerprozesses} waren ebenso Voraussetzung, wie das Testen der Software mittels der Erlang-Module \textbf{fprof} und \textbf{cprof}. Das Endprodukt sollte die üblichen Vorkommnisse im bargeldlosen Zahlungsverkehr darstellen können, der Sicherheitsaspekt wurde bei der Bearbeitung des Projekts jedoch nicht berücksichtigt.
